%%%
%%% packages
%%%

\usetheme{breakout}
\usepackage[utf8]{inputenc}
\usepackage{amsmath,amsfonts,amssymb,amsthm,bm,mathtools}
\usepackage{pgfplots, tikz}
\usetikzlibrary{calc}
\usepackage{graphicx}
\usepackage{anyfontsize}
\usepackage{xcolor}
\usepackage{algorithm2e}
\usepackage{hyperref}
\usepackage[capitalise]{cleveref}

%%%
%%% plotting
%%%

\graphicspath{{figures/}}
\pgfplotsset{compat=1.14}
\usetikzlibrary{arrows.meta,
                bending,
                intersections,
                quotes,
                shapes.geometric}

%%%
%%% variables
%%%

\def \tighten {0.2cm}

%%%
%%% beamer template
%%%

\setbeamertemplate{theorems}[numbered]
\newcommand{\semitransp}[2][35]{\color{fg!#1}#2}

% parskip
\setlength\parindent{0pt}
\setlength\parskip{6pt}

%%%
%%% theorems
%%%

\theoremstyle{plain}
\newtheorem{assumption}{Assumption}
\newtheorem*{assumption*}{Assumption}

%%%
%%% hyperlinks etc
%%%

\hypersetup{
	linktoc=all,
	colorlinks=true,
	linkcolor={magenta},
	filecolor=magenta,
	urlcolor=magenta,
	citecolor={green!50!black},
}

%%%
%%% math operators
%%%

\DeclareMathOperator*{\argmin}{arg\,min}
\DeclareMathOperator{\Hess}{Hess}
\DeclareMathOperator{\vspan}{span}
\DeclareMathOperator{\diver}{div}

%%%
%%% custom symbols
%%%

% << x >> double angle norm
\DeclareFontFamily{OMX}{MnSymbolE}{}
\DeclareFontShape{OMX}{MnSymbolE}{m}{n}{
    <-6>  MnSymbolE5
   <6-7>  MnSymbolE6
   <7-8>  MnSymbolE7
   <8-9>  MnSymbolE8
   <9-10> MnSymbolE9
  <10-12> MnSymbolE10
  <12->   MnSymbolE12}{}
\DeclareSymbolFont{mnlargesymbols}{OMX}{MnSymbolE}{m}{n}
\SetSymbolFont{mnlargesymbols}{bold}{OMX}{MnSymbolE}{b}{n}
\DeclareMathDelimiter{\llangle}{\mathopen}{mnlargesymbols}{'164}{mnlargesymbols}{'164}
\DeclareMathDelimiter{\rrangle}{\mathclose}{mnlargesymbols}{'171}{mnlargesymbols}{'171}

%%%
%%% table of contents
%%%

\AtBeginSection[]
{
\setbeamertemplate{section in toc shaded}[default][40]
\begin{frame}<beamer>
  \frametitle{Outline}
  \tableofcontents[currentsection]
\end{frame}
}